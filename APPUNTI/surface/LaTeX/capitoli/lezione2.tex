\section{12-03-25: | Thermodynamics}

All surfaces are energetically unfavourable, since the have a positive free energy. This is because they are obtained by breaking bonds. So to increase a surface the energy required will be:
\begin{equation}
    -dw_s = dG_S = \gamma d\sigma 
\end{equation}
\begin{equation}
    \gamma_{hkl} = \left(\frac{\partial G_S}{\partial \sigma}\right)_{P,T} = \left[\frac{J}{m^2}\right]
\end{equation}
Which can be related to the surface tension, so an energy that contrasts the generation of new surface.

\paragraph{Sulfactants:} They decrease the surface tension of a surface. Which is not the natural tendency of systems. For example water tends to generate spheres to reduce the surface. The difficulty in removing an atom from the surface is the fact that the atoms try to get inside the bulk, where energies are lower. 

In crystals we don't have spheres, this is because the forces applied to the atoms are anisotropic. We can try with adsorption of a gas phase and finally altering the local surface to decrease the energy. Just remember that surfaces with higher density are the more stable ones.
\\In order of stability:
\begin{itemize}
    \item (111) FCC
    \item (100) BCC
\end{itemize}
But according to just this we should find just octahedron, since they are made only of (111). But remember that we also want to minimize the surface, this means that cutting it (creating a nano-sheet / bipiramid) we mantain a main (111) structure and a (100) just to reduce the area.

Based on different values of surface tension, we will obtain different geometrical forces.
And this can be done by using a sulfactant that will affect just certain types of surface directions.

\paragraph{Alterating structure:} Relaxation and reconstruction. The first one is very low energy and just changes the interatomic distances by a bit. Reconstruction on the other hand changes the periodicity near the surface. The change happens after the cut, so we can observe the before and after.

\paragraph{Gas adsorption:} It always occours in the atmosphere. There's difference between adsorption (molecules DO NOT enter the material) and absorption (molecules enter the material). From the formula, we obtain the 'heat of adsorption' given from the stability increase of the system.