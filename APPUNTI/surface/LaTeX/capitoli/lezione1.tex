\section{11-03-25: | Introduction}
\paragraph{What is a surface?} It is a boundary between 2 different phases, there's a change of density, but it is a continues transition. The interaction between the two are just all the possible scenarios where one phase changes into the other. It is not a sudden change.

\paragraph{Porosity:} We can different types of pores, follow the slides (\textit{pag. 10}). These are all the possible construction.

\paragraph{Pores:} Classification (\textcolor{red}{Important for exam}):
\begin{itemize}
    \item Micropores: $w < 2nm$
    \item Mesopores: $2nm \leq w \leq 50nm$
    \item Macropores: $w < 50nm$
\end{itemize}
Possible applications are adsorption and filtration.

\paragraph{Why surfaces?} Catalysis, micro/nanoelectronics, Energy, Biological (brain, photosynthesis). Remember that there's a gigantic gap between the experimental conditions and than the practical applications.