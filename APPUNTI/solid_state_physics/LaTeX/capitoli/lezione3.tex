\section{13-03-25: | Continuation of Tight model (Forneris)}

When we consider functions that are very broad, they can sum such that the $\ket{\varphi_+}$ almost creates a continuous line, which means that's as if the $\lambda \rightarrow \infty$. But for the rest nothing changes, the periodicity stays the same nonetheless.

\paragraph{1° Brillouin zone:} This is the smallest zone of the crystal where i can define the physical properties that describe the properties of all the crystal. It is defined in the reciprocal space because everything ranges from:
\begin{equation}
    k \in \left[0, \frac{\pi}{d}\right]
\end{equation}
Where the range indicates the frequency range assumed by $k$, so 0 is the $\varphi_+$ and the other the negative wavefunction.
We can take a step further and calculate the $\Delta E_{\pm}$ which means the maximum difference in energy for both the + version and the - version.

\begin{equation}
    \Delta E_{\pm} = \int dx \braket{\varphi_{\pm}|\hat{V}_{\text{pert}}|\varphi_{\pm}}
\end{equation}
The question is which delta energy will be greater? We must consider that the + one, has a smaller excursion, but the higher area covered. So in conclusion, yeah, the $\Delta E_+ < \Delta E_-$. 

If we compare and plot the $\Delta E$ with the interatomic distance, we notice the difference between the 2 tends both to 0, this is because at extreme distances, there are no overlaps. While as long as they differ, we have an energy band, indicating the \textit{Difference between the maximum excursion of energy generated from the different frequency of the wavefunction.}. It generates the pictures we studied with Scarano, where we have the continues variation of energy when travelling inside the Brillouin zone.

\subsection{Direct \& Indirect lattice}

\paragraph{Direct / Bravais:} In the direct lattice we have the position of the nodes and they can be reached by using traslation vectors multiplied an integer number of times.
\paragraph{Indirect / Reciprocal:} All representad using the wave vector $\vec{k}$, all the positions can again be reached by using a complex number times the base vectors.

\paragraph{Primitive cell:} Smallest amount of space needed to describe the whole crystal by just applying a range of traslations. You can see some examples / exercises in the Obsidian file. Next follows a representation of energy bands, DOS and points of itnerest.