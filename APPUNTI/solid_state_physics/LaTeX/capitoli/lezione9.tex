\section{25-03-25: Direct \& Reverse bias | (Olivero)}

\textit{Recap of the last lecture.}

\paragraph{Biased junction:} Remember that we use a coordinate system where we apply +1V from the source to the p part first thus reducing the $\Delta V$ in the junction.

So when we apply a POSITIVE voltage, the BIAS voltage will DECREASE: $V_{bi} \rightarrow V_{bi} - V$, so that the $\Delta V$ will diminuish. The opposite will happen changing a sign or the origin of the voltage.

\paragraph{Capacitance:} We can measure the electrical capacitance: let's say we are in an asymmetrical doping, so that $N_A \gg N_D \rightarrow N_D / N_A \ll 1$. By mantaining the charge neutrality, $x_p \ll x_n$. This will change the triangle plot, by changing the size of each zone. It will reduce the $x_p$ region and/or enlarge the $x_n$ zone.

Now, remember the formula for the capacitance for 2 planes:
\begin{equation}
    C = \frac{Q}{V} \qquad \text{More general definition} \; \rightarrow \qquad C = \frac{\partial Q(V)}{\partial V}  
\end{equation}
So lets start from the charge Q:
\begin{equation}
    Q = \underbrace{(eN_D)}_{vol. \; density} \;\cdot \underbrace{\underbrace{S}_{surface} \cdot \underbrace{\sqrt{2eN_D \varepsilon_0\varepsilon_r ( V_{bi} + V)}}_{width}}_{volume}
\end{equation}

\begin{equation}
    C = \underbrace{S \cdot \sqrt{\frac{eN_D\varepsilon_0\varepsilon_r}{2(V + V_{bi})}}}_{derivative\;of\;Q}
\end{equation}
We can at the end linearize the formula by plotting $1/C^2$ instead of $C$, getting us something like:
\begin{equation}
    \frac{1}{C^2} = aV + b
\end{equation}
This lets us know the value experimentally $N_D$ from $a$:
\begin{equation}
    N_D = \frac{2}{S^2e\varepsilon_0\varepsilon_rQ}
\end{equation}
Once we have $N_D$, we also can determine $b$, which is the $V_{bi}$.

\paragraph{Applied V:} What happens when we apply a voltage to the junction?
\begin{equation}
    V_{bi} - V = \frac{k_BT}{e}\ln\left(\frac{p_{p}}{p_{n}}\right)
\end{equation}
Notice that we don't have $p_{p0}$ and $n_{n0}$, because we are not in initial conditions. What will happen is that the amount of holes in $x_n$ will increase the lower the $\Delta V$ gets.
In this conditions we can say that the concentration of $p_{p0}$ and $p_{p}$ is almost equal, but there's a big change in $p_n$.

\paragraph{Weak injection conditions:} 
\begin{equation}
    p_n = p_{n0}\underbrace{\exp\left(\frac{eV}{k_BT}\right)}_{V>0\; \rightarrow\; ris<1}
\end{equation}

\begin{itemize}
    \item Reverse bias: $p_n < p_{n0}$ 
    \item Direct bias: $p_n > p_{n0}$
\end{itemize}
Far from the depletion zone, we have only diffusion at equilibrium.

\paragraph{Continuity equation:} 
To describe the current that passes in direct bias near the depletion zone we use the continuity equation:
\begin{equation}
    \frac{\partial p_p}{\partial t} + \frac{\partial j_p}{\partial x} = - q U
\end{equation}
Where $U$ is the net recombination rate. Normally it is 0, but in direct bias the generation has an additional term:
\begin{equation}
    G_{new} = G +\Delta G
\end{equation}
Which are the carriers moved in from the $\Delta V$.

\paragraph{Diffusion Lenght $L_P$:} Obsidian files.
