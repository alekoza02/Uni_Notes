\section{24-03-25: |  (Forneris)}

EM waves in materials:
\begin{equation}
    \underbrace{\vec{p}}_{\text{Polarization field}} = \underbrace{n_c}_{\text{density of micr. dipols}} \quad\cdot\quad \underbrace{\hat{p}}_{\text{micr. dipole moment}}
\end{equation}
Displacement field:
\begin{equation}
    \vec{D} = \varepsilon_0 \mathcal{E} + \vec{P}
\end{equation}
Now we can rewrite the Displacement field as:
\begin{equation}
    \vec{D} = \varepsilon_0 \varepsilon_r \mathcal{E}
\end{equation}
We also can introduce:
\begin{equation}
    \vec{H} = \frac{\vec{B}}{\mu_0\mu_r}
\end{equation}
And lastly rewrite the Maxwell equations (previous section).
We can then use the Ohm's law to express how the charges are moving in the material:
\begin{equation}
    V = RI
\end{equation}
\begin{equation}
    \vec{j} = \underbrace{\sigma}_{\text{conductivity}} \vec{\mathcal{E}}
\end{equation}
As an anticipation, this conductivity will depend on the frequency of the EM. So the dipols of the material will oscillate when interacting with it, but also the opposite happens: the dipols will change the EM.

So to proceed we can combine two equations and get:
\begin{equation}
    \vec{\nabla} \left( \vec{\nabla} \vec{\mathcal{E}}\right) = - \frac{\partial}{\partial t} \left(\vec{\nabla} \times \vec{B}\right)
\end{equation}
\textit{Continua una dimostrazione sulle wave equations for materials, fai riferimento ai file obsidian. It seems that the result will be similar to the description of the prism and rainbows. We obtain at the end the refractive index. Lambert-Beer law.}

\subsection{Free electrons - plasma oscillation}
Using Drude's model we can have a point charge when we apply a static electric field.   
