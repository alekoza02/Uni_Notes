\section{28-03-25: Interactions of bound electrons with light | (Forneris)}

\begin{equation}
    \eta = \sqrt{\left(\varepsilon_r + \frac{i\sigma(\omega)}{\varepsilon_0\omega}\right)}
\end{equation}
The first term in the parenth. is the polarizability and the second describes the free electron. So in free electron, it oscillates due to the energy of the wave, what about the bound-electron?
It still oscillates, but now that is bound, it has a descrete set of energy in which it can move. So only certain amount of energy in the waves can be absorbed. This explanation is based on the Tight-Binding model, where $e^-\ket{\varphi}$.

We now consider the wavefunction of the crystal perturbated by the EMW, so we now have:

\begin{equation}
    \hat{H}_{\text{em}}(t) =  \hat{H}_{\text{crystal}} + \hat{V}_{\text{em}}(t)
\end{equation}
\begin{equation}
    \ket{\Phi(x,t)} = \frac{1}{B}\sum_{E_l} b_e(l)\ket{\Phi_e(l)}
\end{equation}
This says that based on the probability of the jump occuring, energy and so on we can calculate the probability of the electron to be found in an excited state.

\begin{equation}
    \ket{\Psi(x, t)} = \frac{1}{B} \left[b_1(t) \ket{\Phi_1 (x)} + b_2(t) \ket{\Phi_2 (x)}\right]
\end{equation}
We now considered the most basic case, but this equation can describe all the possible transition, even the recombination (low), not only jump (high).

Unfortunately, we can't use the stationary Schr\"{o}dinger, so we will use the time evolution:
\begin{equation}
    i\hbar\frac{\partial}{\partial t} \ket{\Phi(x,t)} = \hat{H}_{elm}(t)\Phi(x, t)
\end{equation}
To solve it follow the steps:
\begin{itemize}
    \item Plus the equation $\ket{\hat{H}_{elm}\Psi(x,t)}$ in the Schr\"odinger equation's.
    \item Find the expression relation for ${b_l(t)}$
    \item Assumptions on $\hat{V}_{elm}$ strenght $(\hat{V}_{elm} \ll \hat{V}_{crystal})$
\end{itemize}
Now we need to connect it to refractive index:
\begin{equation}
    \vec{D} = \varepsilon_0\varepsilon_r \vec{\mathcal{E}} = \varepsilon_0 \mathcal{E} + \underbrace{\vec{P}}_{Polarization}
\end{equation}
\begin{equation}
    \vec{p} = \varepsilon_0 (\varepsilon_r - 1) \vec{\mathcal{E}}
\end{equation}
\begin{equation}
    \varepsilon_r = 1 + \frac{\vec{p}}{\varepsilon_0 \vec{\mathcal{E}}} 
\end{equation}
\begin{equation}
    \vec{p} = n_{b\vec{p}}
\end{equation}
So it's the density of micropoles generated by bound electrons.

\vspace{15pt}

\dots

\vspace{15pt}
\begin{equation}
    \varepsilon_r(\omega) = 1 + \sum_{l\neq G}{F_l\frac{\hbar^2\omega^2_b}{(E_l - E_G)^2 - \hbar^2\omega^2}}
\end{equation}
Where the G is the ground state, l is the higher energy state and $F_l$ us the strenght of the interaction (selection rule).
The denominator is the resonance equation. Get the conclusions from Obsidian.

\subsection{Ligh scattering}
We have:
\begin{itemize}
    \item Elastic: delayed emission ($10^{-12} - 10^{-7}$s) is the principle one (Rayleigh scattering)
    \item Inelastic: $h\nu \neq h\nu'$
\end{itemize}