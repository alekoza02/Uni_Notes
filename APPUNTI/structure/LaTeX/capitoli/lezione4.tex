\section{18-03-25: | Reconstructing crystal structure}

\begin{equation}
    I_{\text{Theor.}} (\bar{H}) \sim |F(\bar{H})|^2 \frac{V_{\text{cry}}}{V_{\text{EC}}^2}K(2\theta)
\end{equation}
By comparing experimental and theoretical value, we find that it is necessary to add a scale factor that takes into account all the factors of the real world that can change the value.
\begin{equation}
    I_{\text{Exper.}} (\bar{H}) \sim \underbrace{|F(\bar{H})|^2 \frac{V_{\text{cry}}}{V_{\text{EC}}^2}K(2\theta)}_{\text{Depends on the crystal}} \qquad\cdot\underbrace{\mathcal{S}_{cale}}_{\text{Depends on the setup}}
\end{equation}
We can obtain $\Omega$, which is the resulting sum of all the different errors generated by the approximation of out mathematical model. If we try to minimize $\Omega$, we obtain the most optimal setup.
\begin{equation}
    \Omega = \sum_{\bar{H}} \frac{\left[I(\bar{H})_{\text{Theo}} - I(\bar{H})_{\text{Expe}}\right]^2}{\sigma^2(\bar{H})}
\end{equation}
We can change the position, the scale, the $B_j$ factor to approximate better our crystal model, also known as \textit{`Structure refinement'}.

\subsection{Structure solving}

We can start from examining the electron density (invariant in the lattice), obtaining the Fourier coefficients that gives us structure factors and \textit{hkl}. The problem is that the Intensity is proportional to the square of the structure factor, and this one is a complex funciton, so we encounter the 'phase problem', because we don't know how to calculate it. There are 3 methods to solve this:

\paragraph{MEM (Maximum Entropy Method):} We consider which funciton maximize the entropy of the function:
\begin{equation}
    \Omega = \sum_{j=1}^{Pixels} \rho_j \ln(\rho_j) = \text{Entropy}
\end{equation}
This is calculated for each point in our lattice, meaning we try to calculate the best possible electron density, which is the one whose value (totally unknown) maximize the previous expression.
We also have some CONSTRAINTS:
\begin{itemize}
    \item ${\rho_j} \rightarrow F_{\text{Theo}}(\bar{H})$: You must know the number of electrons in the elementary cell (you know the composition of the crystal).
    \item $|F_{\text{Theo}}(\bar{H})| = |F_{\text{Exp}}(\bar{H})|$: You must know that the factor structure has the same modulus (you don't know the phase tho).
\end{itemize}

\paragraph{Charge flipping:} I assume that the modulus is equal, but i still don't know the phase, so what i do is to take a guess function that change sign each time it is negative. Considering this as a new factor structure i repeat everything again until I reach the correct rappresentation of the crystal structure.
\paragraph{Direct methods:} It's a set of methods with a common basic idea, but we will not investigate them.

\subsection{Anomalous scattering} We start from the Larmor Law:
\begin{equation}
    \mathcal{P} \propto q^2 \bar{a}^2 
\end{equation}