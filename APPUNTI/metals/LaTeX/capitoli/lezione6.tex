\section{20-03-25: | Nucleation theory}

The important quantities are free energy and surface energy. The nucleation rate depends on two factors:
\begin{itemize}
    \item Smaller the temperature, the higher the faster the growth.
    \item Higher the temperature, the higher the number of nucleous.
\end{itemize}
These two terms combine giving the maximal solidification speed where their product is the highest. Changing the composition, will change this temperature. In heterogeneous nucleation we have the interface of a boundary. So it's more convienient for nucleation to start there, since removing that interface with a lower energy volume is thermodinamically more convienient. It will also depend on the angle, as was shown in previous lessons, where the energy depends on the orientation of the grains. A coherent interface will also be favored, so once again, since creating a new nucleation center costs energy but it's lower than the boundary grains it happens. So it starts almost always following the grain structure (coherent) and then expands chaotically (incoherent).

During the nucleation there's a diffusivity and segregation, since we are on a boundary, there is stress caused by the surface and defects. When the nucleation happens, atoms are favored to move and reorganize themselves. So depending on the heterogeneous phase we will have a diffusion of a single phase from $\gamma$ to a pure phase $\alpha$ or $\beta$.

We can additionally tell that coherent nucleous will be planar, while incoherent will be curved.

\paragraph{Growth rate:} $v = \partial x / \partial t$, while the concentration of a certain pure phase will diminsh overtime, until it's completely depleted. Before the nucleation we have an homogeneous solution and concentrations. And after that the regions around the nucleous will loose atoms and accumulate them at the nucleous. We can define it as:
\begin{equation}
    J = D \frac{dC}{dx}
\end{equation}
We can model it using for example a basic model called `Zener model'. Basically it models a simple diffusion using rectangles and triangles, than it smooths it out by substituting with more complex fuctions and integrating. What this model lacks is a growth stop, it doesn't consider when crystals starts interacting with themselves.