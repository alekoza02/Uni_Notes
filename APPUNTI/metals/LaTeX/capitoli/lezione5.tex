\section{18-03-25: | Phase transformations in solids}

They happen among solid phases and they require diffusion. Using thermodynamic we can classify them as:
\begin{itemize}
    \item First order: Discontinuity in first order derivative, there's latent heat (solidification)
    \item Second order: Discontinuity in second order derivative, there's a step in specific heat (magnetic transition)
\end{itemize}
The diffusion transformations involve long range movement of atoms, which can be further divided in heterogeneous (nucleation and growth) and homogeneous (spinodal decomposition(?)). In the end we have Displacive Transformations which involve shift of atoms but over relatively short distances (martensitic transformation).

\paragraph{Classification:} 
\begin{itemize}
    \item a) precipitation: small precipitation when the solubility limit is surpassed
    \item b) eutectoid: a single solid state that transforms into two different solid states upon cooling 
    \item c) ordering reaction: a rearrange without actually changing the metal phase, small increase of internal stability
    \item d) massive: a reaction so fast that it "skip" intermediate transformations, leading to a clear change from parent phase to stable child phase
    \item e) polymorphic: same as massive transformation, but for pure elements or single compounds
\end{itemize}

\paragraph{Nucleation:} In ideal case we have a spherical nucleation where the energy to create a small cluster will depend on surface energy and stability in the bulk. 
There are more informations, but they are all equal to Rizzi's lectures. 