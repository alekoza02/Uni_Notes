\section{31-03-25: | Eutectoid transformation}

Using the TTT curve, we can relate the Temperature to the log(T) which gives us the information of when and how long a certain transformation takes. If we have more than one microstructure, they can overlap. This is the case for Bainite and Pearlite for example. If we cool even more, we change the phase (into martensite)

\subsection{Bainite}
Bainite contains ferrite and cementite.
\paragraph{High Ferrite:} The ferrite forms between 2 grains of austenite along the $\perp$ of the grain border (along a direction of the grain), rejecting the carbon.
\paragraph{High Cementite:} Generation of cones of ferrite with carbides needle on them.

\paragraph{Civilian / Military:} When you find these terms, they are referred on how the atoms move: civilian is like a crowd, chaotic and on average zero. Military is ordered, in a specific order.

\subsection{Addition of alloying elements:} TTT curves is shifted to longer times, beacuse alloying materials stabilize some phases:

\begin{itemize}
    \item Austenite: Mn, Ni, Cu. They lower the eutectoid temperature, slowing the diffusion
    \item Ferrite: Cr, Mo, Si. They increase the eutectoid temperature, which promotes individual structures (high ferrite) if there's a partition.
    \item Carbide: Cr, Mo, Mn. If there's no partitioning, these elements 'poison' ferrite nucleation sites with carbides.
\end{itemize}
In the notes there are various plots showing how 'noses' are formed or removed due to alloying elements.

\subsection{CCT}

First of all they are shifted to lower temperature than TTT, they indicate how fast you cool your system, the intersection tells you what kind oh phases and microstructures you will encounter.

\subsection{Quenching}

Discussion of spinodal range, if we mantain the temperature, the solution can diffuse and generate two different regions of different composition, which are more favorable in free energy. They do not change, but they lower the total free energy. This can exagerate until you meet the limits of the miscibility gaps. BUT if the starting concentration is outside the spinodal range, the free energy increases (in the beginning), stopping this process with an energetic barrier (nucleation energy). In the previous case there was no nucleation.  

\paragraph{Massive transformation:} It is a big jump in free energy, going from a phase $\beta$ to $\alpha$ for example and than precipitating as usual. The massive part is related to the big change from the 2 initial phases. the transformation occours via migration, but with a very high jump in energy. 