\section{17-03-25: | Interfaces in metallurgy}

\begin{itemize}
    \item Free surfaces: between metal and liquid or vapour
    \item Grain boundaires: between same phase ($\alpha/\alpha$)
    \item Interphase Interfaces: between different phase ($\alpha/\beta$)
\end{itemize}

The surface tension is a phenomenon that happens in liquids only (stress free surfaces). In the metals we have the surface energy, which is a measure of the excess energy at the surface of a material compared to its bulk. This also gives information about the energy needed to create one unit of surface. Its like the work needed to increase or decrease the surface.
\begin{equation}
    dW_{s(T,P) = \gamma dA}
\end{equation}
s(T,P) means constant in temperature and preassure. We also have surface / interface stress, which the energy required to deform reversebly a unit on area.

\paragraph{Interfaces:} Sharp and diffuse, they can change the atomic configuration in a long or short range. The surface phase changes at every single atomic layer, because the related Gibbs dividing surface changes. This is a mathematical trick to explain the gradient encountered in the interfaces. In general the greater the number of boundaries, greater is the strain resistance, since making the plane atoms slide and encounter boundaries requires more energy to deform.

Grains with smaller angle than 15° are called low-angle grain boundaries. The energy that will build up at the boundaries, is the total energy of all dislocations piled up. There's a strain and tension due to additional atomic planes, which will create a surface (difference of pressure and density). Here's way the low angle boundaries have less energy: there are less dislocations.

When the angle suprasses 15° the energy required flattens, even at higher angles there's the same energy requirement. This is beacuse they are so different that the grains do not try to form a boundary formed by dislocations. They just have some smaller interactions at a long range (couple of atomic distances).

\paragraph{Special cases:}
It might happen that due to symmetry of the cell orientation, we can get a \textit{Site coincidence}, where it's asif we have a low angle boundary. 

\noindent Another case is th \textit{Twin boundaries}, where at specific angles we got a coherent (or incoherent) twinning plane where a boundary actually coincise with another plane.

So when dealing with 2 different phases we can actually have the same cases: coherent, semi-coherent and incoherent.

\paragraph{Laplace-Young equations:} It describes the effect on interfaces at the equilibrium.
\begin{equation}
    \Delta P = \gamma \left(\frac{\partial A}{\partial V}\right)_T
\end{equation}
This describes also how the temperature changes at different size of particles, this is beacuse of the Gibbs-Thomson effect, which says that small particles have higher free energies (lower melting point).