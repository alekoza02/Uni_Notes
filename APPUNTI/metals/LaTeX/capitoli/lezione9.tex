\section{01-04-25: | Martensitic transformation}

\subsection{Finishing last lecture}
\paragraph{Short range order:}
We may have ordering solutions, where the alloy arranges itself with a certain order and stechiometry, which can be expressed with long and short range. This of course can range from complete disorder to complete order.
We can use the `short range parameter':
\begin{equation}
    s = \frac{P_{AB} - P_{AB}(random)}{P_{AB}(max) - P_{AB}(random)}
\end{equation}
Where $P_{AB}$ means the number of bonds between $A-B$. 
\paragraph{Long range order:} Occours only at lower temperatures, ensures a certain order over long distances. Has the following dependecy:
\begin{equation}
    L = \frac{r_A - X_A}{1 - X_A}
\end{equation}
Where $r_A$ is the probability that A is on the right sublattice, namely it is in the correct ideal lattice.

\subsection{Martensite}

Forms essentially from Austenite $(\gamma)$ and is usually called Martensite $(\alpha')$ and it's a BCT (Tetragonal).

This transformation happens very quickly, so generally the order is at short range.
There are some invariant planes, where the local atoms do not move relatively to the surroundings.
What happens is that the crystallographic structure (bain strain) we don't have anymore an invariant plane (due to non-uniform scaling), unless we apply a rotation to increase the lattice correspondance.
What happens in reality tho, is that Martensite creates many planes that shear and create many deformations that pile up.

The lateral growth is mainly preferred, here exaplained why `lens-like' structures form.
Further heat treatment can help remove brittleness.

The shape of the lens structures depends on the amount of carbon, and also the habit plane orientation, this is due to the crystalline distorsion of the system. 

\subsection{Solidification}

\begin{itemize}
    \item Casting: quello che vedi di solito nei reels.
    \item Welding: classical method, create sheets and then joining them together.
    \item Additive manufacturing (3D printing): new production method, extremely precise microstructure.
\end{itemize}

\paragraph{Nucleation:} The good news is that we move from a liquid to a solid phase, this means we don't have strain contribution in the formation energy.
